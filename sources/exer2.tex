\documentclass[a4paper,oneside,10pt]{book}

%These tell TeX which packages to use.
\usepackage[utf8]{inputenc}
\usepackage[english,russian]{babel}
\usepackage{array,epsfig}
\usepackage{amsmath}
\usepackage{amsfonts}
\usepackage{amssymb}
\usepackage{amsxtra}
\usepackage{amsthm}
\usepackage{mathrsfs}
\usepackage{color}

\usepackage{caption,multicol}
\setlength{\columnsep}{0.2cm}
\usepackage{pdfpages}


%Here I define some theorem styles and shortcut commands for symbols I use often
\theoremstyle{definition}
\newtheorem{defn}{Definition}
\newtheorem{thm}{Theorem}
\newtheorem{cor}{Corollary}
\newtheorem*{rmk}{Remark}
\newtheorem{lem}{Lemma}
\newtheorem*{joke}{Joke}
\newtheorem{ex}{Example}
\newtheorem*{soln}{Solution}
\newtheorem{prop}{Proposition}


%Pagination stuff.
\setlength{\topmargin}{-.3 in}
\setlength{\oddsidemargin}{0in}
\setlength{\evensidemargin}{0in}
\setlength{\textheight}{9.in}
\setlength{\textwidth}{6.5in}
\pagestyle{empty}



\begin{document}


\begin{center}
{\large  ТАУ \hspace{0.1cm} Лабораторная работа \textnumero 2}

\vspace{5pt}
\textit{\large Временные и частотные характеристики}\\ %You should put your name here
\vspace{10pt}
Due: 4 апреля 2020 %You should write the date here.
\end{center}

\vspace{0.2 cm}



Пусть система управления с одним входом~$ x(t) $ и одним выходом~$ y(t) $ задана  передаточной функцией~$ W(s) $: 

\begin{multicols}{2}

	
	\begin{enumerate}
		\item
		
		$ 	W(s) = 
		\dfrac{s+1}
		{(s+2)(0.04s^2 + 0.2s+1)} $;
		\item 
		
		$ 	W(s) = 
		\dfrac{2(s+2)}
		{(s+1)(0.09s^2+0.3s+1)} $;
		
		\item 
		
		$ 		W(s) = 
		\dfrac{3(s+1)}
		{(s+3)(0.16s^2+0.4+1)} $;
		
		\item
		
		$ 	W(s) = 
		\dfrac{4(s+3)}
		{(s+1)(0.25s^2+0.5s+1)} $;
		
		\item 
		
		$ 	W(s) = 
		\dfrac{5(s+3)}
		{(s+1)(0.36s^2+0.6s+1)} $;
		
		\item
		
		$ 	W(s) = 
		\dfrac{6(s+4)}
		{(s+1)(0.49s^2+0.7s+1)} $;
		
		\item
		
		$ W(s) = 
		\dfrac{7(s+4)}
		{(s+2)(0.64s^2+0.8s+1)} $;
		
		\item
		
		$ W(s) = 
		\dfrac{8(s+5)}
		{(s+3)(0.25s^2+0.7s+1)} $;
		
		\item 
		
		$ W(s) = 
		\dfrac{9(s+5)}
		{(s+2)(0.16s^2+0.56s+1)} $;
		
	
		
		
		
		\item
		
		$ 	W(s) = 
		\dfrac{2(s+1)}
		{(s+3)(0.49s^2+0.7s+1)} $;
		
		\item
		
		$ W(s) = 
		\dfrac{5(s+7)}
		{(s+1)(0.64s^2+0.8s+1)} $;
		
		\item
		
		$ W(s) = 
		\dfrac{8(s+4)}
		{(s+2)(0.25s^2+0.7s+1)} $;
		
		\item 
		
		$ W(s) = 
		\dfrac{9(s+1)}
		{(s+1)(0.16s^2+0.56s+1)} $;
		
		
		
		
		\item 
		
		$ W(s) = 
		\dfrac{10(s+5)}
		{(s+4)(0.36s^2+0.84s+1)} $.
		
		
	\end{enumerate}
\end{multicols}


\subsection*{\textit{Задания}}

\begin{enumerate}
\item
Найти переходную и импульсную функции. 
\item \label{ordinary_eq}
Найти АЧХ и ФЧХ системы. 
Определить реакцию системы в установившемся режиме при входном сигнале~$ x(t) = 2\sin(0.5t) $.

\item 
Используя пакет control найти: 
\begin{itemize}
	\item
	переходную функцию (\textit{step}), 
	\item
	импульсную функцию (\textit{impulse}),
	\item
	реакцию на входной сигнал~$ x(t) = 2\sin(0.5t) $ (\textit{forced-response}),
	\item 
	характеристики выходного сигнала в установившемся режиме (\textit{freqresp}),
	\item
	построить диаграмму Найквиста. 
\end{itemize}

Сравнить полученные результаты. 
\end{enumerate}



\end{document}


