\documentclass[a4paper,oneside,10pt]{book}

%These tell TeX which packages to use.
\usepackage[utf8]{inputenc}
\usepackage[english,russian]{babel}
\usepackage{array,epsfig}
\usepackage{amsmath}
\usepackage{amsfonts}
\usepackage{amssymb}
\usepackage{amsxtra}
\usepackage{amsthm}
\usepackage{mathrsfs}
\usepackage{color}

\usepackage{caption,multicol}
\setlength{\columnsep}{0.2cm}
\usepackage{pdfpages}


%Here I define some theorem styles and shortcut commands for symbols I use often
\theoremstyle{definition}
\newtheorem{defn}{Definition}
\newtheorem{thm}{Theorem}
\newtheorem{cor}{Corollary}
\newtheorem*{rmk}{Remark}
\newtheorem{lem}{Lemma}
\newtheorem*{joke}{Joke}
\newtheorem{ex}{Example}
\newtheorem*{soln}{Solution}
\newtheorem{prop}{Proposition}


%Pagination stuff.
\setlength{\topmargin}{-.3 in}
\setlength{\oddsidemargin}{0in}
\setlength{\evensidemargin}{0in}
\setlength{\textheight}{9.in}
\setlength{\textwidth}{6.5in}
\pagestyle{empty}



\begin{document}


\begin{center}
	{\large  ТАУ \hspace{0.1cm} Лабораторная работа \textnumero 1}

	\vspace{5pt}
	\textit{\large Передаточные функции}\\ %You should put your name here
	\vspace{10pt}
	Due: 28 марта 2020 %You should write the date here.
\end{center}

\vspace{0.2 cm}



Пусть система управления с одним входом~$ x(t) $ и одним выходом~$ y(t) $ задана  передаточной функцией~$ W(s) $:

\begin{multicols}{2}
	\begin{enumerate}
		\item

		      $ 	W(s) =
			      \dfrac{2(s+1)}
			      {(s+2)^2(s+3)} $;
		\item

		      $ 	W(s) =
			      \dfrac{4(s+2)}
			      {(s+3)^2(s+1)} $;

		\item

		      $ 		W(s) =
			      \dfrac{6(s+2)}
			      {(s+4)^2(s+4)} $;

		\item

		      $ 	W(s) =
			      \dfrac{8(s+5)}
			      {(s+3)^2(s+4)} $;

		\item

		      $ 	W(s) =
			      \dfrac{10(s+4)}
			      {(s+3)^2(s+5)} $;

		\item

		      $ 	W(s) =
			      \dfrac{3(s+2)}
			      {(s+1)^2(s+3)} $;

		\item

		      $ W(s) =
			      \dfrac{5(s+4)}
			      {(s+1)^2(s+1)} $;
		\item

		      $ W(s) =
			      \dfrac{4(s+2)}
			      {(s+3)^2(s+1)} $;

		\item

		      $ W(s) =
			      \dfrac{7(s+2)}
			      {(s+2)^2(s+4)} $;

		\item

		      $ 	W(s) =
			      \dfrac{9(s+3)}
			      {(s+5)^2(s+4)} $;

		\item

		      $ W(s) =
			      \dfrac{8(s+2)}
			      {(s+1)^2(s+5)} $;

		\item

		      $ W(s) =
			      \dfrac{8(s+2)}
			      {(s+8)^2(s+5)} $;

		\item

		      $ W(s) =
			      \dfrac{3(s+1)}
			      {(s+7)^2(s+2)} $;

		\item

		      $ W(s) =
			      \dfrac{2(s+2)}
			      {(s+8)^2(s+4)} $.

	\end{enumerate}
\end{multicols}


\subsection*{\textit{Задания}}

\begin{enumerate}
	\item
	      Выписать дифференциальное уравнение по передаточной функции~$ W(s) $.
	\item \label{ordinary_eq}
	      Найти решение дифференциального уравнения, полагая~$ x(t) = 0 $, при некоторых начальных условиях.
	\item \label{unordinary_eq}
	      Определить переходную функцию системы ($ x(t) =1 $, с нулевыми начальными условиями), используя для этого методы теории дифференциальных уравнений, операторные методы.
	\item
	      Для исходных уравнений из пунктов~\ref{ordinary_eq}, \ref{unordinary_eq} найти численное решение методом Эйлера. Сравнить полученные результаты с аналитическим решением, построив графики и определив максимальную относительную погрешность.  Подобрать такой шаг в методе Эйлера, чтобы максимальная относительная погрешность не превышала 0,01\%.


\end{enumerate}



\end{document}


