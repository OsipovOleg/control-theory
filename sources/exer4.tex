\documentclass[a4paper,oneside,10pt]{book}

%These tell TeX which packages to use.
\usepackage[utf8]{inputenc}
\usepackage[english,russian]{babel}
\usepackage{array,epsfig}
\usepackage{amsmath}
\usepackage{amsfonts}
\usepackage{amssymb}
\usepackage{amsxtra}
\usepackage{amsthm}
\usepackage{mathrsfs}
\usepackage{color}

\usepackage{caption,multicol}
\setlength{\columnsep}{0.2cm}
\usepackage{pdfpages}


%Here I define some theorem styles and shortcut commands for symbols I use often
\theoremstyle{definition}
\newtheorem{defn}{Definition}
\newtheorem{thm}{Theorem}
\newtheorem{cor}{Corollary}
\newtheorem*{rmk}{Remark}
\newtheorem{lem}{Lemma}
\newtheorem*{joke}{Joke}
\newtheorem{ex}{Example}
\newtheorem*{soln}{Solution}
\newtheorem{prop}{Proposition}


%Pagination stuff.
\setlength{\topmargin}{-.3 in}
\setlength{\oddsidemargin}{0in}
\setlength{\evensidemargin}{0in}
\setlength{\textheight}{9.in}
\setlength{\textwidth}{6.5in}
\pagestyle{empty}



\begin{document}
	
	
	\begin{center}
		{\large  ТАУ \hspace{0.1cm} Лабораторная работа \textnumero 4}
		
		\vspace{5pt}
		\textit{\large Устойчивость и качество систем упроавления}\\ %You should put your name here
		\vspace{10pt}
		Due: 16 мая 2020 %You should write the date here.
	\end{center}
	
	\vspace{0.2 cm}
	
	
	
	Пусть система управления с одним входом~$ x(t) $ и одним выходом~$ y(t) $ задана  передаточной функцией~$ W(s) $: 
	
	\begin{equation*}
			W(s) = 
		\dfrac{a_1 s+1}
		{b_3 s^3 + b_2 s^2 + b_1 s +1} , 
	\end{equation*}
	
	где коэффициенты  определяются согласно варианту $ n $: 
	\begin{equation*}
		a_1 = n, \quad b_1 = n, \quad b_2 = (n\bmod 3) + 1, \quad b_3 = (n \bmod 8) + 1.  
	\end{equation*}
	
	\subsection*{\textit{Задания}}
	
	\begin{enumerate}
		\item
		Определить усточивость системы, используя основной критерий устойчивости, критерий Гурвица. Найти степень устойчивости. 
		\item
		Определить область устойчивости для системы, зафиксировав только значение $ b_1 $.  
		
		\
		
		
		\item 
		Для   $(a_1, b_3, b_2)  \in \mathcal{D}$ из области устойчивости, полученной в задании 2, определить следующие характеристики качества систем управления (используя пакет control) (варьируем каждую компоненту вектора минумум 4 раза, строим графики для каждого параметра в одной системе координат): 
		\begin{itemize}
			\item
			время переходного процесса $ t_{\text{п}} $,
			\item
			максимальное отклонение в переходный период (перерегулирование $ \sigma $),
			\item
			колебательность переходного процесса.
			
		\end{itemize}
		
		Сравнить полученные харатеристики, сделать выводы. 
	\end{enumerate}
	
	
	
\end{document}


